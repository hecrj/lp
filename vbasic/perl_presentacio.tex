\documentclass{beamer}
\usepackage[utf8]{inputenc}
\usepackage[usenames,dvipsnames]{color}
\definecolor{LighterGray}{rgb}{0.9,0.9,0.9}
\definecolor{LightGray}{rgb}{0.7,0.7,0.7}
\definecolor{DarkGray}{rgb}{0.4,0.4,0.4}

\newcommand{\cita}[1]{%
  \begin{flushright}%
    \tiny{[ #1 ]}%
  \end{flushright} %
}

\usepackage{listings}
\lstset{ %
  language=Perl,%
  basicstyle=\ttfamily\footnotesize,
  breakatwhitespace=false,%
  commentstyle=\color{DarkGray},%
  breaklines=true,%
  keywordstyle=\color{blue},%
  showspaces=false,
  showstringspaces=false,
  showtabs=false,
  numbers=left,
  numbersep=5pt,
  numberstyle=\tiny\color{LightGray},
  stringstyle=\color{orange},
  escapeinside=||
}

%% aplicacions -> Popularitza les regexp, Projecte Genoma Humà, cgi (perl al web)
%% multiparadigma -> ensenyar el map, sobretot imperatiu, OOP ->MOOSE
%% sistema de tipus-> tipus bàsics  @ % $ & *
%% format lliure
%% curiositats -> concepte de context (moltes vegades costa llegir),
%% s'acostuma a dir que es illegible,  moto TMTOWTDI. les crides a funcions no cal parentitzar-les (si no hi ha ambiguitat)

%% closures

\usetheme{Madrid}
%\usecolortheme{beaver}

\setbeamertemplate{navigation symbols}{}
\setbeamertemplate{footline}[page number]{}

\defbeamertemplate*{title page}{customized}[1][]
{
  \begin{center}
    \begin{beamercolorbox}[sep=10pt,center]{title}
      \usebeamerfont{title}\inserttitle
    \end{beamercolorbox}
    \hfill \\[0.5cm]
    \usebeamercolor[fg]{titlegraphic}\inserttitlegraphic
    \\[0.5cm]
    \usebeamerfont{author}\footnotesize\insertauthor
    \\[0.2cm]
    \usebeamerfont{date}\tiny\insertdate
  \end{center}
}
\title{Perl}
\titlegraphic{\includegraphics[scale=0.15]{perl_camel_logo.png}}
\author{Alvaro Espuña}
\date{19 desembre 2013}

\begin{document}
\frame{\titlepage}

\begin{frame}{Introducció}
  \begin{description}
    \item[Quan?] 1987
      % Tot i que s'ha modernitzat força, arrossega coses encara.
    \item[Qui?] Larry Wall (lingüista)
      % Notar que té coses curioses gràcies a que el tio és lingüistax
    \item[Per què?] AWK no és suficient i C és massa. \\[0.1cm]
      \emph{Perl is designed to make the easy jobs easy, without
        making the hard jobs impossible.}  \cita{\textbf{Larry Wall},
        \emph{Programming Perl}}
      % La idea és ser un llenguatge d'scripting, però potent
    \item[Influencia] PHP, Python
  \end{description}
\end{frame}

\begin{frame}[fragile]{Característiques}
  \begin{description}
  \item[Scripting] \hfill
    \begin{itemize}
    \item[·] \texttt{\#!/usr/bin/perl}
    \item[·] Interpretat en dues fases, compilació i execució.
    \item[·] Llenguatge cohesionant, molt útil per a tasques administratives.
    \end{itemize}
    \hfill \\[0.5cm]
    \begin{center}
      \begin{minipage}{0.6\textwidth}
    \begin{lstlisting}
my @dades = qw[./programa1];
@dades = processar @dades;
qw[./programa2 @dades];
    \end{lstlisting}
    \end{minipage}
      \end{center}
  \end{description}
\end{frame}
\begin{frame}{Característiques}
  \begin{description}
  \item[Multiparadigma] \hfill
    \begin{itemize}
    \item[·] Inicialment únicament procedural, influenciat per C.
    \item[·] Amb Perl 5 es \emph{modernitza}:
      \begin{description}
        \item[Funcional] \hfill
          \begin{itemize}
            \item[-] Funcions d'ordre superior
            \item[-] $\lambda$ expressions (\texttt{\textbf{sub} \{...\}})
            \item[-] \emph{Closures} i \emph{currying} \\
          \cita{\textbf{Mark Jason Dominus}, \emph{Higher Order Perl}}
          \end{itemize}
        \item [OOP] \hfill
          \begin{itemize}
            \item[-] \emph{Duck typing}
            \item[-] Herència múltiple (diferents \textbf{mro})
          \end{itemize}
      \end{description}
    \end{itemize}
  \end{description}
\end{frame}

\begin{frame}{Característiques}
  \begin{description}
    \item[Gestió de la memòria] \hfill
      \begin{itemize}
        \item[·] Automàtica.
        \item[·] Utilitza un gc que compta referències.
        \item[·] No allibera immediatament quan surt de l'\emph{scope}
          \cita{\textbf{perlfaq3} How can I free an array or hash so my program shrinks?}
      \end{itemize}
  \end{description}
\end{frame}

\begin{frame}[fragile]{Característiques}
  \begin{description}
    \item[Tipat dèbil] \hfill \\
      \begin{lstlisting}
use strict;
use warnings;
print "a" + 1; # 1
      \end{lstlisting}
      {\footnotesize\texttt{Argument "a" isn't numeric in addition (+) at - line 3.}}
  \item[Tipat dinàmic] \hfill \\
    Els sígils determinen el \emph{tipus} de les variables.
  \item[Àmbit] \hfill \\
    L'àmbit de visibilitat de les variables ve determinat pels blocs
    (\texttt\{...\}) i per les paraules clau \textbf{my} i
    \textbf{our}.
  \end{description}
\end{frame}

\begin{frame}[fragile]{Tipus de dades bàsics}
  \begin{description}
    \item[Escalars, Arrays, Hashes (\texttt{\$}, \texttt{@}, \texttt{\%})] \hfill\\
      \begin{lstlisting}
my $nombre = 1.42 + 1_234_567_890;
my $text = "Hola!\n";
# Llistes
my @array = (1, 2, 3);
my $primer = $array[0];
my $mida = @array;
# Associative arrays d'AWK
my %hash = (
   poma => "vermella",
   kiwi => "verd"
);
my $color = $hash{'poma'};
my @fruites = keys %hash;
      \end{lstlisting}
  \end{description}
\end{frame}

\begin{frame}[fragile]{Tipus de dades bàsics}
  \begin{description}
    \item[Funcions (\texttt{\&})] \hfill\\
      \begin{lstlisting}
sub my_map {
    (my $funcio, my @llista) = @_;
    my @ret;
    foreach (@llista) { # for $_ in @llista
        push @ret, $funcio->($_);
    }
    return @ret; # No caldria, return implicit
}
      \end{lstlisting}
      Crida:
      \begin{lstlisting}
map sub { 2*shift }, (1,2,3,4); # (2, 4, 6, 8)
map \&f, (1,2,3,4); # sub f {...} anterior
      \end{lstlisting}
  \end{description}
\end{frame}

\begin{frame}[fragile]{Expressions regulars}
  \indent Perl va popularitzar les expressions regulars (PCRE). \\
  \indent Estan completament integrades dins el llenguatge.\\
  \begin{center}
  \begin{minipage}{0.75\textwidth}
  \begin{lstlisting}
# Escriure tot excepte les linies en blanc
while(<STDIN>) { print unless (/^\s*$/); }
  \end{lstlisting}
  \end{minipage}
  \end{center}
\end{frame}
\begin{frame}[fragile]{Expressions regulars}
  Operadors \texttt{m//}, \texttt{s///}:
  \begin{center}
  \begin{minipage}{0.75\textwidth}
  \begin{lstlisting}
# Matching
$text =~ m/abc/; # el mateix que /abc/
$text =~ /\/\/\//; # el mateix que m[///];
$text =~ /abc/i; # case-insensitive
# Substitucions
$text =~ s/mantega/oli/;
$text =~ s/mantega/oli/g; # global
# Captura $0, $1, $2,...
$text =~ m/^([0-9]+)-[0-9]+$/;
$prefix = $1;
$text =~ s/\b(p[^\s]+)/uc($1)/eg;
  \end{lstlisting}
  \end{minipage}
  \end{center}
\end{frame}

\begin{frame}{Aplicacions}
  \begin{description}
    \item[Common Gateway Interface] \hfill \\
      Tot i que cada cop PHP i Python tenen més tirada:
      \pause
       \begin{itemize}
         \item[-] Bugzilla
         \item[-] Craigslist
         \item[-] IMDb
         \item[-] Slashdot
         \item[-] ...
       \end{itemize}
     \item[Tractament de \emph{strings}] \hfill \\
         Perl va ser clau en el \emph{Human Genome Project}
  \end{description}
\end{frame}

\end{document}
